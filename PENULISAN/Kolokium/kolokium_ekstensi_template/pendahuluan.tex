%----------------------------------------------------------------------------------------
%	PENDAHULUAN
%----------------------------------------------------------------------------------------

\section*{PENDAHULUAN} % Sub Judul PENDAHULUAN
% Tuliskan isi Pendahuluan di bagian bawah ini. 
% Jika ingin menambahkan Sub-Sub Judul lainnya, silakan melihat contoh yang ada.
% Sub-sub Judul 

\subsection*{Latar Belakang}
Subsektor peternakan memiliki peranan penting dalam perekonomian Indonesia baik dalam pembentukan Produk Domestik Bruto (PDB) dan penyerapan tenaga kerja maupun dalam penyediaan bahan baku industri. Perannya dalam pertumbuhan ekonomi menunjukkan bahwa PDB peternakan triwulan I tahun 2005 tumbuh 5.8\%. Kontribusi PDB subsektor peternakan terhadap sektor pertanian triwulan I tahun 2005 mencapai 13.2\%. Sedangkan terhadap besaran PDB Nasional mencapai 2\%. Dalam penyerapan tenaga kerja sub sektor peternakan juga mempunyai peranan yang sangat strategis. Menurut hasil sensus pertanian 2003 dari 24,86 juta Rumah Tangga Pertanian di pedesaan dan perkotaan, sekitar 22,63\% merupakan Rumah Tangga Usaha Peternakan. Selain itu sub sektor peternakan juga berperan penting dalam penyediaan bahan baku bagi keperluan industri (\cite{Makka2012}). 

Efisiensi produksi dalam suatu usaha peternakan menjadi faktor penentu keberhasilan peternakan. Efisiensi produksi dapat diwujudkan dengan pemberian pakan yang berkualitas dengan kuantitas yang memadai sesuai dengan kebutuhan ternak. Pakan merupakan salah satu aspek yang sangat penting dalam keberhasilan suatu usaha peternakan. Sehingga formulasi ransum dari sejumlah bahan pakan yang tersedia merupakan aspek yang sangat vital khususnya dalam rangka menyeimbangkan kandungan energi, protein dan nutrien lainnya (\cite{Jayanegara2014}). Berdasarkan sudut pandang ekonomi, biaya untuk pembelian pakan ternak merupakan biaya tertinggi dalam agribisnis perternakan. Sehingga biaya tersebut harus ditekan serendah mungkin agar tidak mengurangi pendapatan. Teknologi dapat menjadi jalan keluar dalam permasalahan tersebut, yaitu dengan mengaplikasikan teknologi formulasi pakan ternak yang efisien. Pakan ternak yang diramu dengan baik dan sesuai dengan kebutuhan ternak akan menekan biaya pembelian pakan serendah mungkin (\cite{Shiddieqy2010}). 

Ransum yang murah dan berkualitas memerlukan suatu teknik atau metode formulasi ransum yang mudah digunakan, cepat, akurat dalam penentuan komposisi bahan (perhitungan) dan mendapatkan biaya serendah mungkin dalam perhitungannya. Metode formulasi tersebut adalah metode pemrograman linier. Selain metode pemrograman linier, ada beberapa metode lain yang dapat digunakan, antara lain metode \textit{trial and error}, \textit{equation} dan \textit{pearson’s square}. Diantara metode-metode tersebut, metode pemrograman linier adalah yang paling sesuai untuk diterapkan sebagai metode formulasi ransum karena harga ransum dapat dimasukkan sebagai peubah (fungsi tujuan) dalam perhitungan, akan tetapi dalam perhitungannya secara menual metode ini masih dirasa sangat sulit (\cite{Kusnandar2004}).

Penelitian tentang formulasi ransum ternak sapi sudah pernah dilakukan oleh \citeauthor{Rahman2017} (\cite*{Rahman2017}). Peneliti membuat sistem formulasi ransum berbasis web dengan batasan hewan ternak sapi potong. Sistem tersebut dapat melakukan formulasi dengan kesamaan dan akurasi yang baik karna hasil perbandingan mendapatkan selisih 0. Penelitian lainnya juga pernah dilakukan oleh \citeauthor{Muzayyanah2013} (\cite*{Muzayyanah2013}) dalam pembuatan sistem pakar formulasi pakan unggas menggunakan \textit{linier programming} pada sistem berbasis \textit{mobile}. Penelitian ini menggunakan metode pengembangan sistem \textit{prototype}. Hasil dari penelitian ini adalah sebuah sistem pakar yang mampu menghasilkan ransum dengan harga yang lebih murah.

Sehingga penelitian ini mengadopsi kelebihan dari dua penelitian sebelumnya. Pada penelitian ini diusulkan sebuah sistem pendukung pengambilan keputusan berbasis \textit{mobile} untuk mendukung formulasi pakan ternak sapi yang mudah digunakan oleh peternak yang lebih sering melakukan aktivitas bergerak. Sistem formulasi ransum ini dibuat untuk mengatur kandungan nutrisi pada pakan ternak sapi berdasar pada kebutuhan ternak dengan tujuan menekan biaya pakan seminimal mungkin. Fomulasi dilakukan dengan menggunakan metode pemrograman linier. Menurut \citeauthor{Muzayyanah2013} (\cite*{Muzayyanah2013}) metode ini dipilih karena mampu menangani jumlah variabel yang banyak secara efisien . 

% Sub-sub Judul 
\subsection*{Perumusan Masalah}
Berdasarkan uraian yang tercantum pada latar belakang, dapat dirumuskan adanya kebutuhan sistem formulasi ransum yang mudah diakses oleh pengguna dengan aktivitas bergerak yang lebih banyak. Oleh karena itu, dibutuhkan sistem formulasi ransum berbasis \textit{mobile} yang mampu memformulasikan pakan unggas secara cepat dan mudah diakses. Sistem formulasi ini harus mampu menyusun pakan berdasar pada kebutuhan nutrisi yang optimal dengan biaya pakan yang seminimal mungkin.

\subsection*{Tujuan}
Tujuan dari penelitian ini adalah:
\begin{enumerate}[noitemsep] 
\item Mengembangkan sistem formulasi ransum berbasis \textit{mobile} yang mampu menentukan jumlah bahan pakan yang digunakan serta biaya pakan yang dibutuhkan dengan menggunakan metode pemrograman linier.
\item Mempercepat pengguna dalam memformulasikan pakan ternak.
\end{enumerate}

\subsection*{Ruang Lingkup}
Ruang lingkup penelitian adalah:
\begin{enumerate}[noitemsep] 
\item Jenis ternak yang disusun ransumnya yaitu sapi perah dan sapi potong.
\item Sistem dikembangkan pada sistem berbasis web dan \textit{mobile}. 
\end{enumerate}

\subsection*{Manfaat}
Hasil penelitian diharapkan dapat membantu para peternak dalam melakukan formulasi ransum secara cepat dan tepat.

