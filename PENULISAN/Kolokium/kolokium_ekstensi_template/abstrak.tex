%----------------------------------------------------------------------------------------
%	ABSTRACT
%----------------------------------------------------------------------------------------
\Abstract{\scriptsize 
% ---- Tuliskan abstrak di bagian ini seperti contoh.
Formulasi ransum merupakan aspek yang sangat esensial dalam menyeimbangkan nutrisi bagi hewan ternak dengan tujuan mendapatkan harga minimum berdasar pada kandungan nutrisi pakan hewan. Oleh karena itu peternak dituntut untuk mampu menyusun suatu formula ransum yang ekonomis tanpa mengabaikan faktor kebutuhan nutrisi ternak. Penelitian ini bertujuan untuk membuat suatu sistem pendukung pengambilan keputusan yang mampu melakukan formulasi ransum dengan mengadopsi metode pemrograman linier. Sistem dirancang dalam perograman web dan mobile sehingga formulasi ransum dapat dilakukan oleh pengguna di peternakan dan pengolahan data dapat dilakukan menggunakan peramban. Metode pengembangan yang dilakukan adalah \textit{prototype} dengan evaluasi \textit{black-box testing} dan uji pengguna menggunakan perbandingan aplikasi WinFeed 2.8.
\\
\\
Feed formulation is an essential aspect in balancing nutrients for livestock in order to get a minimum price based on the nutrient content of livestock feed.
Therefore, farmers are required to be able to compile an economical feed formulation without ignoring nutritional needs factors of livestock.
This research aims to create/develop a decision support system that is capable of compile feed formulation by adopting the method of linear programming.
The system is designed in web and mobile programming so that the feed formulation can be conducted by users in farms and data processing can be done using a web browser.
The development method used is prototype with evaluation of black-box testing and user test using comparison of WinFeed 2.8 application.
% ---- Akhir bagian abstrak
\normalsize}
