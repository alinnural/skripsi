%----------------------------------------------------------------------------------------
%	ABSTRACT
%----------------------------------------------------------------------------------------
\Abstract{\scriptsize 
% ---- Tuliskan abstrak bahasa Indonesia di bagian ini seperti contoh.
Bertambahnya keanekaragaman tanaman obat menyebabkan dokumentasi hasil penelitian tanaman obat semakin bertambah, sehingga mengakibatkan kesulitan dalam hal pencarian dokumen. Diperlukan suatu sistem pencarian yang dapat menemukembalikan dokumen yang dicari dengan menggunakan kueri. Pada penelitian ini akan dilakukan pengembangan sistem pencarian pada dokumen RDF menggunakan Sesame dan Lucene. Pembobotan menggunakan Tf-idf sebagai nilai relevansi terhadap dokumen yang ditemukembalikan. Penggunaan sistem Lucene sebagai mesin pencari menghasilkan rata-rata precision pada kueri tanpa term boosting sebesar 0.862. Sedangkan untuk kueri yang diberikan term boosting menghasilkan rata-rata precision sebesar 0.877.
% ---- Akhir bagian abstrak bahasa Indonesia
\\%
\\%
\textit{%
% ---- Tuliskan abstrak bahasa Inggris di bagian ini seperti contoh
Increasing the diversity of medicinal plants causing documentation of the results of research medicinal plants was increasing that led to the difficulty in terms of search of a document. Required a searching system that can be retrieves documents using query. In this study, we will develop searching system for RDF documents using Sesame and Lucene. TF-idf used as relevant value about documents retrivied. Query without boosting term generate average precision 0.862 while query with boosting term generate average precision 0.877.
% ---- Akhir bagian abstrak bahasa Inggris
}%
\normalsize}
